\documentclass[onecolumn]{IEEEtran}
\IEEEoverridecommandlockouts
% The preceding line is only needed to identify funding in the first footnote. If that is unneeded, please comment it out.
\usepackage[noadjust]{cite}
\renewcommand\citepunct{, }
%\usepackage[caption=false]{subfig}
\usepackage{subcaption}
\usepackage{amsmath,amssymb,amsfonts}
\usepackage[ruled,vlined,linesnumbered]{algorithm2e}
% \usepackage{algorithm2e}
\usepackage{bm}
\usepackage{graphicx}
\usepackage{enumitem}
\usepackage{makecell}
\usepackage{booktabs}
\usepackage{adjustbox}
\usepackage{textcomp}
\usepackage{color,soul}
\usepackage{multirow}
\usepackage{multicol}
\usepackage{comment}
\usepackage{balance}
\usepackage{tabularx}
\usepackage[yyyymmdd,hhmmss]{datetime}
\usepackage{array}
\usepackage{geometry}
\usepackage{textcase}
\usepackage[tablename=TABLE]{caption}
% \usepackage[colorlinks=true, allcolors=blue]{hyperref}
\usepackage{setspace} % Add to preamble
\usepackage{hyperref}
\def\BibTeX{{\rm B\kern-.05em{\sc i\kern-.025em b}\kern-.08em
		T\kern-.1667em\lower.7ex\hbox{E}\kern-.125emX}}
\usepackage[usenames,dvipsnames]{xcolor}
\usepackage[framemethod=TikZ]{mdframed}
\mdfdefinestyle{default}{%
	linewidth=1pt,
	backgroundcolor=gray!10,
	innertopmargin=10pt,
	innerbottommargin=10pt,
	innerrightmargin=10pt,
	innerleftmargin=10pt,
	roundcorner=4pt
}
\mdfsetup{style=default}
% Setup for compact itemize in tables only
\makeatletter
\newenvironment{tableitemize}{%
\begin{spacing}{0.9}
	\begin{itemize}[
    leftmargin=*,
    topsep=0pt,
    partopsep=0pt,
    parsep=0pt,
	font=\small
  ]}{%
  \end{itemize}\end{spacing}}
\makeatother
% }{\end{itemize}}
% \makeatother
\usepackage{listings}
\lstset{
	%backgroundcolor=\color{backcolour},   
	commentstyle=\color{codegreen},
	keywordstyle=\color{blue},
	numberstyle=\tiny\color{codegray},
	stringstyle=\color{codepurple},
	basicstyle=\footnotesize,
	escapeinside=``,
	breaklines,
	columns=fullflexible,
	identifierstyle={},
	stringstyle=\ttfamily,
	extendedchars=false,
	linewidth=1\textwidth,
	numbers=left,
	numbersep=2 em, 
	tabsize=4,
	frame=leftline,
	showstringspaces=false
}
\usepackage{tikz}
\usepackage{tikz-qtree}
\usetikzlibrary{trees}
\IEEEoverridecommandlockouts

\SetKwInput{KwData}{Input}
\SetKwInput{KwResult}{Output}
\SetKwInOut{Init}{Init}
\newlength\mylena
\newcommand\myinput[1]{%
	\settowidth\mylena{\KwData{}}%
	\setlength\hangindent{\mylena}%
	\hspace*{\mylena}#1\\}
\newlength\mylenb
\newcommand\myoutput[1]{%
	\settowidth\mylenb{\KwResult{}}%
	\setlength\hangindent{\mylenb}%
	\hspace*{\mylenb}#1\\}
\newlength\mylenc
\newcommand\myinit[1]{%
	\settowidth\mylenc{\Init{}}%
	\setlength\hangindent{\mylenc}%
	\hspace*{\mylenc}#1\\}
\newtheorem{theorem}{Theorem}
\begin{document}
	\title{Machine learning-based occupancy grid mapping}
	\author{
		\IEEEauthorblockN{Peiyuan Zhai}
		%\IEEEauthorblockA{5276179}
	}
	
	\maketitle
	\begin{abstract}
		In this note, there are contents and comments on ML-based occupancy grid mapping algorithms. 
		This is the public version of a private copy of the note. It might be updated in the future. 
	\end{abstract}
	% \section{Link to other related notes}
	% \begin{itemize}
	% 	\item \href{https://github.com/hantyou/-Journal-Ver.-Radar-and-lidar-fusion-for-OGM.git}{SBL-OGM sensor fusion journal paper github repo} and \href{https://www.overleaf.com/project/66b4c5fb0ef5b0698669e45a}{overleaf link}
	% \end{itemize}
	\section{Table of works}
	\clearpage

\newgeometry{left=0.25in, right=0.25in, top=0.25in, bottom=0.25in}
\begin{table*}[htbp]
\centering
\caption{Table of Deep Learning-based OGM methods}
\label{tab:dl-ogm}
\begin{adjustbox}{max width=\textwidth}
\newcolumntype{P}[1]{>{\raggedright\arraybackslash}p{#1}}
\newcolumntype{M}[1]{>{\raggedright\arraybackslash}m{#1}}
\newcolumntype{C}[1]{>{\centering\arraybackslash}m{#1}}
\newcolumntype{Y}{>{\raggedright\arraybackslash}X}
% Define column widths using array package
\begin{tabularx}{\textwidth}{
C{0.065\textwidth} % Type
C{0.13\textwidth} % Name
C{0.04\textwidth} % Year
C{0.03\textwidth} % Code
C{0.07\textwidth} % Structure
C{0.055\textwidth} % Input
C{0.05\textwidth} % Output
C{0.14\textwidth} % Comments
P{0.13\textwidth} % Pro
P{0.13\textwidth} % Con
}
\toprule
Type & Name & Year & Code & Structure & Input & Output & Comments & Pro & Con \\
\midrule
\multirow{16}{=}{{Single modal}}
& RadarOcc~\cite{ding_radarocc_2024} & 2024 & {\color{green}\href{https://github.com/Toytiny/RadarOcc}{\checkmark}} & transformer & 4D radar & 3D SSC & 
& \multirow{4}{=}{\begin{tableitemize}
\item DL structure help 4D radar feature learning (i.e., doppler)
\item second point
\end{tableitemize}} & \multirow{4}{=}{ }\\
& Occ3D~\cite{tian_occ3d_2023} & 2023 & {\color{green}\href{https://github.com/Tsinghua-MARS-Lab/Occ3D}{\checkmark}} & transformer & RGB & 3D SSC & & & \\
& VoxelNet~\cite{zhou_voxelnet_2017} & 2017 & {\color{green}\href{https://github.com/skyhehe123/VoxelNet-pytorch}{\checkmark}} & CNN & LiDAR & 3D box & & & \\
& SurroundOcc~\cite{wei_surroundocc_2023} & 2023 & {\color{green}\href{https://github.com/weiyithu/SurroundOcc}{\checkmark}} & transformer & RGB & 3D SSC & & & \\
& DeepISM~\cite{weston_probably_2019}& 2019 & {\color{red}\texttimes} & VAE & radar & 2D OGM & Evidence available& & \\
&Evidential DeepISM~\cite{bauer_deep_2019} & 2019 & {\color{red}\texttimes} & VAE & radar & 2D (E)OGM & Evidence available& & \\
&\hl{PointPillars}~\cite{lang_pointpillars_2019}& 2019 & {\color{green}\href{https://github.com/nutonomy/second.pytorch}{\checkmark}} & CNN & LiDAR & 3D box & & & \\
&OccupancyNet~\cite{sless_road_2019} & 2019 & {\color{red}\texttimes} & CNN & radar BEV & 2D OGM & & & \\
&\makecell{SemanticPON~\cite{roddick_predicting_2020}}& 2020 & \makecell{{\color{green}\href{https://github.com/tom-roddick/mono-semantic-maps}{\checkmark}}} & STPN & \makecell{RGB} & 2D SSC & & & \\ 
&\makecell{MotionNet~\cite{wu_motionnet_2020}}& 2020 & \makecell{{\color{green}\href{https://github.com/pxiangwu/MotionNet}{\checkmark}}} & STPN & \makecell{LiDAR} & 2D OGM & & & \\
&EviLOG~\cite{van_kempen_simulation_based_2021}& 2021 & {\color{green}\href{https://github.com/ika-rwth-aachen/EviLOG}{\checkmark}} & CNN & radar & 2D EOGM & & & \\
&\hl{RPFA-Net}~\cite{xu_rpfa_net_2021}& 2021 & {\color{green}\href{https://github.com/Nerdmust/astyx-pcdet-radar.git}{\checkmark}} & CNN & radar & 2D OGM & & & \\
\cline{1-2}
\multirow{4}{=}{Single sensor augmented}
& & & {\color{red}\texttimes} & & & & & & \\
& & & {\color{red}\texttimes} & & & & & & \\
& & & {\color{red}\texttimes} & & & & & & \\
& & & {\color{red}\texttimes} & & & & & & \\
\cline{1-2}
\multirow{4}{=}{Multi-sensors}
& \makecell{MVDNet~\cite{qian_mvdnet_2021}\\ST-MVDNet~\cite{li_st_mvdnet_2022}\\ST-MVDNet++~\cite{li_st_mvdnetpp_2023}} & \makecell{2021\\2022\\2023} & \makecell{{\color{green}\href{https://github.com/qiank10/MVDNet}{\checkmark}}\\{\color{red}{\texttimes}}\\{\color{red}{\texttimes}}} & CNN & radar LiDAR & 2D box & & \multirow{4}{=}{\begin{tableitemize}\item Multi-sensor fusion\item s\end{tableitemize}} & \multirow{4}{=}{\begin{tableitemize}\item High computational cost\item Complex architecture\end{tableitemize}}\\
& MV3D~\cite{chen_multi_view_2017} & 2017 & {\color{green}\href{https://github.com/bshao001/MV3D}{\checkmark}} & CNN & LiDAR camera & 3D box & & & \\
& DTW~\cite{dequaire_deep_2018} & 2018 & {\color{red}\texttimes} & CNN+RNN & LiDAR & 2D OGM+box & & & \\
& AVOD~\cite{ku_joint_2018} & 2018 & {\color{red}\texttimes} & CNN & LiDAR camera & 3D box & & & \\
& FISHING~\cite{hendy_fishing_2020} & 2020 & {\color{red}{\texttimes}} & CNN & radar LiDAR camera & 3D box & Limited semantic, some obstacles are not detected & & \\
&RadarNet~\cite{yang_radarnet_2020} & 2020 & {\color{red}\texttimes} & CNN & radar LiDAR & 3D box & & & \\
\midrule
\multicolumn{10}{l}{\textbf{Note:} SSC: Semantic Scene Completion, OGM: Occupancy Grid Map, EOGM: Evidential OGM}\\
\bottomrule
\end{tabularx}
\end{adjustbox}
\vspace{5pt}
\end{table*}
\clearpage 
\bibliographystyle{ieeetr}
\bibliography{main}
\end{document}